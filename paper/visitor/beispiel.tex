
Um das Visitor-Pattern besser zu demonstrieren, wird dieses anhand des folgenden Beispiels erklärt. In einer Küche gibt es mehrere Sorten Gemüse. Die Variation dieser verschiedenen Sorten ist überschaubar und wird sich nicht mehr ändern. Allerdings ist unklar, welche Operationen auf diese Objekte angewendet werden und in Zukunft noch angewendet werden könnten. Um dieses zu berücksichtigen erstellen wir eine Objektstruktur und verlagern die Operationen auf dieses nicht in den Objekten selbst, sondern außerhalb. Hierfür erstellen wir zunächst ein Interface \texttt{Element} mit der Schnittstelle \texttt{accept(Visitor aVisitor}, der Objekte mit dem Interface \texttt{Vistor} entgegennehmen kann.  Methode.


\begin{listing}[h!]
   \centering
   \javacode{./resources/visitor_element_interface.java}
   \caption{Element Interface}
    \label{visitor_element_interface}
\end{listing}  


Dementsprechend benötigen wir ein Interface \texttt{Visitor} das die Methoden zum Besuchen der einzelnen VArianten der Elemente bereitstellt.

\begin{listing}[h!]
   \centering
   \javacode{./resources/visitor_visitor_interface.java}
   \caption{Visitor Interface}
    \label{visitor_visitor_interface}
\end{listing}  

 
Zunächst betrachten wir einen konkreten Visitor. Die erste Operation für alle Sorten bezieht sich auf das Waschen. Hierfür wird ein WaschenVisitor implementiert, der jeweils alle Methoden bereitstellt zum besuchen des jeweiligen Elements. 

\begin{listing}[h!]
   \centering
   \javacode{./resources/visitor_cleanvisitor.java}
   \caption{CleanVisitor}
    \label{visitor_cleanvisitor}
\end{listing}  


Nachfolgend betrachten wir nur das Element \texttt{Apfel}. In der visit-Methode wird ein Apfel übergeben, der dann auf diesem Element entsprechenden Operationen ausführt.
Diese Methode wird allerdings in der accept-Methode der Klasse \texttt{Apfel} aufgerufen. Die Methode \texttt{accept} ruft die Methode \texttt{visit} des aktuellen Visitors auf (in diesem Fall der WaschenVisitor) und übergibt sich selbst dieser


\begin{listing}[h!]
   \centering
   \javacode{./resources/visitor_potato.java}
   \caption{CleanVisitor}
    \label{visitor_potato}
\end{listing}  


