\subsubsection{Vorteile}
\begin{itemize}
\item es können weitere Operationen hinzugefügt werden, ohne die Objektstruktur anzupassen.
\item Funktionalität kann so gezielt auf bestimmte Arten von Objekten eingesetzt werden.
\item Verwandte Operationen werden im Visitor zentral verwaltet
\item Visitor können mit Objekten aus voneinander unabhängigen Klassenhierachien arbeiten. Mit dem Iterator-Pattern könnten zum Beispiel nur Funktionen aufgerufen werden, die in der Schnittstelle implementiert werden. Beim Iterator-Pattern können auf die verschiedenen Methoden der konkreten Elemente zugegriffen werden.
\end{itemize}
\subsubsection{Nachteile}
\begin{itemize}

\item Der Nachteil ist das durchbrechen der Kapselung. Den Vistors müssen unter Umständen bestimmte Operationen der Elemente zur Verfügung gestellt werden, die den internen Zustand des Objekts verändern.
\item Das Hinzufügen von neuen Elementen ist mit der Änderung von allen Vistors verbunden.
\end{itemize}