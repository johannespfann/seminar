Zur Bearbeitung der Objektstruktur muss jedem Elementen ein Visitor-Objekt über die Accept-Methode übergeben werden. Hierfür benötigt es einen Mechanismus auf diese Elemente zuzugreifen.
Die Literatur (siehe \cite{GOF95} Abschnitt Visitor) stellt drei elegante Varianten zur Traversierung der Objektstruktur vor. Nachfolgend wird erklärt, wie diese aussehen.
\paragraph{Die Objektstruktur} Die Objektstruktur übernimmt die Zuständigkeit, über seine Elemente zu traversieren selbst. Hierfür wird das Composite-Pattern eingesetzt. Dem Root-Element kann man ein Visitor-Objekt über die Accept-Methode übergeben, der diesen weiter verarbeitet und danach seinen Kindelementen übergibt. 
\paragraph{Der Iterator} Eine andere Möglichkeit ist ein interner oder externer Iterator. Dieser könnte bei dem Aufruf des nächsten Elements die Methode \texttt{accept} anstoßen.
\paragraph{Der Visitor} Die letzte der drei Varianten ist, die Traversierung den Visitor-Objekten zu überlassen. Der Nachteil dieser Variante ist , mehrfach den Traversierungscode in den verschiedenen Vistors zu implementieren. Hierfür könnte man allerdings eine abstrakte Klasse einführen, der den redundanten Code eliminiert. Der entscheidende Vorteil dieser Variante ist, verschiedene Möglichkeiten der Traversierung durch die Objektstruktur anzubieten.



