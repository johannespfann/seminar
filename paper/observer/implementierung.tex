\section{Implementierung}
Für das Observer-Pattern gibt es mehrere Implementierungsmöglichkeiten. In Abschnitt 1.1 ... 


\paragraph{Push- oder Pull-Model} Bei den beiden Modellen werden die Unterschiedlichen Arten der Übertragung der Daten betrachtet. Bei dem Push-Model wird bei der Benachrichtigung ein oder mehrere Parameter übertragen. In unserem Fall muss das UML-Diagramm in Abbildung \ref{observerdiagramm} angepasst werden. Die Methode \texttt{aktualisieren()} benötigt einen zusätzlichen Parameter. Dieser wird dann bei dem Objekt \texttt{KonkretesSubjekt} beim Aufruf der Beobachter übergeben. Der Nachteil dieser Variante ist, das im Extremfall mehrere Parameter übergeben werden, jedoch nicht von jeden Observer benötigt werden. Oder ein neuer Beobachter benötrigt andere Daten wodurch man das Interface \texttt{Beobachter} und das Objekt \texttt{KonkretesSubjekt} anpassen muss.
Die Pull Methode verfolgt einen anderen Ansatz. Hier werden keine Daten als Parameter übergeben sondern die Beobachter greifen direkt nach einer Benachrichtigung auf die Instanz \texttt{KonkretesSubjekt} zu um die Daten zu erhalten. Der Nachteil dieser Variante ist, das die einzelnen Beobachter das Subjekt kennen müssen.

%\paragraph{Lagerung der Beobachter} Hintegrund ist folgendes Szenario: Es gibt viele Subjekts und einige Beobachter. Die Subjekts müssen jetzt alle die Instanzen der jeweiligen Beobachter speichern. Um sich diesen Speicher zu ersparen, könnte man die Auflistung aller Beobachter in einer externen Datenstruktur bzw. Objekt lagern.

\paragraph{Beobachter beobachtet mehr als ein Subjekt} Für diesen Fall wird der Beobachter benachrichtigt und kann nicht wissen von welchem Objekt er aufgerufen wird. Abhilfe kann geschaffen werden, indem das konkrete Subjekt seine Instance beim Aufruf übermittelt, sodass der Beobachter eine Fallunterscheidung durchführen kann.

\paragraph{Ausführung des Updates} Das Anstoßen der Benachrichtigungen kann entweder vom Client als auch vom konkreten Subjekt durchgeführt werden. Der Vorteil dies dem Konkretem Subjekt zu überlassen ist die Vermeidung von Fehlern, indem der Client keine Benachrichtigung triggert. Falls die Verantwortung dem Client überlassen wird, kann dieser jedoch besser das Update nach Notwendigkeit steuern.

%\paragraph{Aufräumen der Instansen} Es sollte Beachtet werden, das wenn ein Beobachter sich vom konkreten Objekt entfernt, auch die Instanz des konkreten SUbjekt aufräumt. 

%\paragraph{Vor dem Benachrichtigen das Subjekt zuerst updaten} Bei der Implementierung des konkreten Subjekts sollte beachtet werden, vor der Benachrichtigung der Beobachter, zuerst das Subjekt zu aktualisieren. Gerade bei vererbten Methoden besteht Gefahr eines solchen Szenario.

\paragraph{Erweiterung der Benachrichtigung für ein bestimmtes Interesse} In bestimmten Fällen kann es sein, das ein Beobachter nur auf bestimmte Events benachrichtigt werden möchte. In diesem Fall bietet sich an,  die Methode \texttt{registriere()} um einen weiteren Parameter zu erweitern, damit der Beobachter dadurch das Interesse auf ein bestimmtes Event signalisieren kann.

\paragraph{Auslagerung der Verwaltung der Beobachter} Wenn die Beziehung zwischen den konkreten Subjekten und den Beobachtern zu komplex wird, empfiel sich die Verwaltung dieser auszulagern. In der Literatur wird von einem ChangeManager gesprochen. Dieser regelt folgende Gebiete: Das Mapping zwischen den Subjekten und den Beobachtern. Definiert spezielle Updatestrategien. Übernimmt die Benachrichtigung eines konkreten Subjekts für dessen Beobachter.

%\paragraph{Subjekt als Interface oder Abstrakte Klasse} Das Subjekt kann ein Interface oder als abstrakte klasse implementiert sein. Der Nachteil bei einer abstrakten Klasse ist, dass nicht auf verschiedene konkrete Subjekts eingegangen werden kann. Jedoch erhöht sich der PRogrammieraufwand durch die wiederholte Implementierung des Selben Codes.


