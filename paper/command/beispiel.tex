Das Command-Pattern wird anhand eines Lieferdienstes näher beschrieben. Ein Lieferdienst hat grundsätzliche die Aufgabe Sachgegenstände zu versenden. Welche Gegenstände das sind, bzw. an wen diese gesendet werden, ist losgelöst von der Absicht etwas zu versenden. Möchten wir dieses Szenario implementieren, hätten wir drei Teile: Den Lieferdienst, eine Ausführung und einen Empfänger. Das Command-Pattern gliedert genau diese drei Akteure und stellt als erstes die Ausführung als Interface dar, nämlich dem \texttt{Command} (siehe Abbildung \ref{command_interface}). Diese enthält eine Methode \texttt{execute()} die dann ein Invoker, in unserem Fall der \texttt{DeliveryService} aufrufen kann. 


\begin{listing}[h!]
   \centering
   \javacode{./resources/command_interface.java}
   \caption{Command Interface}
    \label{command_interface}
\end{listing}  

In Listing \ref{command_deliveryservice} wird der Aufruf des DeliveryService demonstriert. Dieser kennt nur das Interface Command, welches die Execute-Methode besitzt. Welche konkrete Implementierung eines Commands in der Methode \texttt{sendObject(Command aCommand)} übergeben und dann ausgeführt wird, ist für den DeliveryService nicht ersichtlich. 

\begin{listing}[h!]
   \centering
   \javacode{./resources/command_DeliveryService.java}
   \caption{DeliveryService}
    \label{command_deliveryservice}
\end{listing} 

Danach werden verschiedene konkrete Commands implementiert, die einerseits den Empfänger kennen und andererseits wissen, wie sie diesen beliefern müssen. Listing \ref{command_deliverbigpackagecommand} zeigt exemplarisch eine Implementierung eines Commands. 


\begin{listing}[h!]
   \centering
   \javacode{./resources/command_DeliverBigPackageCommand.java}
   \caption{DeliverBigPackageCommand}
    \label{command_deliverbigpackagecommand}
\end{listing}  

In Listing \ref{command_main} wird das Zusammenspiel der Akteure dargestellt. Zunächst wird ein DeliveryService initialisiert, danach zwei konkrete Commands. Das Erste ist das \texttt{DeliverBigPackageCommand} das einerseits dem Empfänger und andererseits, für unser Beispiel speziell zugeschnitten, ein Objekt Package übergeben wird. Dem Zweiten, das \texttt{DeliverLetterCommand} wird ebenfalls ein Empfänger und ein zusätzliches Objekt übergeben. Beide Commands implementieren unterschiedliche Schritte zur Ausliefern eines Objektes, können jedoch beide auf diese Art von dem DeliveryService versendet werden. (Listing \ref{command_main} Zeile 10-13).

\begin{listing}[h!]
   \centering
   \javacode{./resources/command_main.java}
   \caption{Main}
    \label{command_main}
\end{listing}  
