\section{Implementierung}
Für das Command-Pattern ist aus der Sicht des Autors folgende drei Erweiterungsmöglichkeiten des Command-Patterns entscheident. Alle drei stammen aus [GoF]. 

\paragraph{Erweiterung durch eine Undo-Funktion} Da jetzt jeder Befehl bzw. Aktion in einem Objekt gekapselt ist, kann man sehr einfach diese in einer Liste oder ähnlichem Lagern. Mit dieser Erkenntnis könnte man auf diese Art eine Undo-Funktion realisieren, die alle getätigten Befehle zurücknimmt um zum Ausgangszustand zurückzukommen. Man könnte hierfür auch das Command-Objekt derart erweitern, dass zusätzliche Informationen

\paragraph{Aufgaben des Command-Objekts} Die Frage die man sich stellen sollte ist: Wie intelligent soll ein Command-Objekt sein. Einerseits kann man alle aufgaben an den Receiver deligieren. Das andere Extrem muss das Command-Objekt nichts von dem Receiver kennen und implementiert die komplette Logik.

\paragraph{Makro-Befehle} Denkbar ist auch, mehrere Receiver an das Command-Objekt zu geben um so mehrere Aktionen durchführen zu können.


