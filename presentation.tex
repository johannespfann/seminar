\documentclass[compress]{beamer}

\mode<presentation>
{
\useoutertheme[subsection=false,footline=authorinstitutetitle]{miniframes}
\useinnertheme{rectangles}
\usecolortheme{whale}
\usecolortheme{orchid}
\usefonttheme{professionalfonts}

\setbeamertemplate{footline}
{
	\begin{beamercolorbox}[wd=0.33\textwidth,ht=2.2ex,dp=0.8ex,leftskip=1.4em,rightskip=1.4em]{author in head/foot}% 
		\usebeamerfont{author in head/foot}%
		\insertshortauthor%
	\end{beamercolorbox}%
 	\vspace*{-3.0ex}\hspace*{0.33\textwidth}%
 	\begin{beamercolorbox}[wd=0.33\textwidth,ht=2.2ex,dp=0.8ex,left,leftskip=1.4em,rightskip=1.4em]{author in head/foot}% 
     	\usebeamerfont{institute in head/foot}%
 		\insertshortinstitute%
 	\end{beamercolorbox}%
  	\begin{beamercolorbox}[wd=0.34\textwidth,ht=2.2ex,dp=0.8ex,left,leftskip=1.4em,rightskip=1.4em]{title in head/foot}
      	\usebeamerfont{title in head/foot}
  		\insertshorttitle
  		\hfill\insertframenumber\,/\,\inserttotalframenumber
  	\end{beamercolorbox}
}

\beamertemplatenavigationsymbolsempty
\setbeamercovered{transparent}

}


\usepackage[ngerman]{babel}
\usepackage[utf8]{inputenc}

% font definitions, try \usepackage{ae} instead of the following
% three lines if you don't like this look
\usepackage{mathpazo}
\usepackage[scaled=.90]{helvet}
%\usepackage{helvet}
\usepackage{courier}
\usepackage{graphics}
\usepackage{amsmath}
\usepackage{amsfonts}
\usepackage{amssymb}


\usepackage[T1]{fontenc}
\usepackage{multicol}


\newcommand{\textqt}[1]{\frqq #1\flqq\ }


\hypersetup{
   pdftitle={Seminarvortrag},
   colorlinks=false,
   linkcolor=black,
   pdfpagemode=None,
   pdfstartview=Fit
  }


\author[Pfann]{%
  Johannes Pfann
}


\date{}

\institute[FAU Erlangen-Nürnberg]{
  Lehrstuhl für Software Engineering\\
  Friedrich-Alexander-Universität Erlangen-Nürnberg
}

\subject{Seminar Design Patterns und Anti-Patterns}
\title{Vortragsthema}

\begin{document}

\frame{\titlepage} 

\frame{\tableofcontents}


\section[Verhaltensmuster]{Verhaltensmuster}

\subsection{Erzeugungsmuster - Was ist das?}
\begin{frame}
	\frametitle{Erzeugungsmuster -- Was ist das?}
	\begin{block}{Erzeugungsmuster ...}
	\begin{itemize}
		\item klassifizieren die Erzeugung von Objekten
		\item kapseln die konkreten Klassen eines Systems
		\item verbergen konkrete Erzeugungsprozesse dieser Klassen	
	\end{itemize}
	\end{block}	
	\begin{block}{Konsequenzen}
	\begin{itemize}
		\item Erhöhte Flexibilität \textit{wer wann was wie} erzeugt.
		\item System wird unabhängiger von der Zusammenstellung und Erzeugung der verwendeten Objekte
		\item Es wird lediglich mit den abstrakten Schnittstellen gearbeitet
	\end{itemize}
	\end{block}	
\end{frame}

\subsection{Typen von Erzeugungsmustern}
\begin{frame}
	\frametitle{Typen von Erzeugungsmustern}
	\begin{block}{Klassenbasiert}
		Bei klassenbasierten Mustern wird Vererbung verwendet
		\begin{itemize}
			\item Factory Method
		\end{itemize} 	
	\end{block}

	\begin{block}{Objektbasiert}
		Bei objektbasierten Mustern wird der Erzeugungsprozess an andere Objekte delegiert
				\begin{itemize}
					\item \alert<2-> {Singleton}
					\item \alert<2-> {Prototype}
					\item \alert<2-> {Abstract Factory}
					\item Builder				
				\end{itemize}	
	\end{block}
\end{frame}

\section[Observer]{Observer}



\section[Command]{Command}



\section[Visitor]{Visitor}



\section[Zusammenfassung]{Zusammenfassung}



\section[Quellen]{Quellen}



\end{document}

%%% Local Variables: 
%%% mode: latex
%%% TeX-master: t
%%% End: 
