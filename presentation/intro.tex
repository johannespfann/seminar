\subsection{Erzeugungsmuster - Was ist das?}
\begin{frame}
	\frametitle{Erzeugungsmuster -- Was ist das?}
	\begin{block}{Erzeugungsmuster ...}
	\begin{itemize}
		\item klassifizieren die Erzeugung von Objekten
		\item kapseln die konkreten Klassen eines Systems
		\item verbergen konkrete Erzeugungsprozesse dieser Klassen	
	\end{itemize}
	\end{block}	
	\begin{block}{Konsequenzen}
	\begin{itemize}
		\item Erhöhte Flexibilität \textit{wer wann was wie} erzeugt.
		\item System wird unabhängiger von der Zusammenstellung und Erzeugung der verwendeten Objekte
		\item Es wird lediglich mit den abstrakten Schnittstellen gearbeitet
	\end{itemize}
	\end{block}	
\end{frame}

\subsection{Typen von Erzeugungsmustern}
\begin{frame}
	\frametitle{Typen von Erzeugungsmustern}
	\begin{block}{Klassenbasiert}
		Bei klassenbasierten Mustern wird Vererbung verwendet
		\begin{itemize}
			\item Factory Method
		\end{itemize} 	
	\end{block}

	\begin{block}{Objektbasiert}
		Bei objektbasierten Mustern wird der Erzeugungsprozess an andere Objekte delegiert
				\begin{itemize}
					\item \alert<2-> {Singleton}
					\item \alert<2-> {Prototype}
					\item \alert<2-> {Abstract Factory}
					\item Builder				
				\end{itemize}	
	\end{block}
\end{frame}