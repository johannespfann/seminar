\subsection*{Erzeugungsmuster - Was ist das?}
\begin{frame}
	\frametitle{Verhaltensmuster -- Was ist das?}
	\begin{block}{Verhaltensmuster ...}
	\begin{itemize}
		\item Zuweisung von Zuständigkeiten
		\item Wechselseitige Kommunikation zwischen Objekten
		\item Beschreibung von komplexen Programmabläufen
	\end{itemize}
	\end{block}	
	\begin{block}{Konsequenzen}
	\begin{itemize}
		\item Trennung von Verantwortlichkeiten
		\item Veringerung der Kopplung 
		\item Erhöhung der Flexibilität der Software hinsichtlich ihres Verhaltens
		\item Bessere Verständlichkeit eines Programmablaufes
	\end{itemize}
	\end{block}	
\end{frame}

\subsection*{Typen von Erzeugungsmustern}
\begin{frame}
	\frametitle{Typen von Erzeugungsmustern}
	\begin{block}{Klassenbasiert}
		Klassenbasierte Verhaltensmuster wenden für die VErhaltenszuordnung zu den Klassen das Vererbungsprinzip an. 
		\begin{itemize}
			\item Template Method
			\item Interpreter
		\end{itemize} 	
	\end{block}

	
\end{frame}

\begin{frame}
	\frametitle{Typen von Erzeugungsmustern}
	\begin{block}{Objektbasiert}
		Bei objektbasierten Mustern wird der Erzeugungsprozess an andere Objekte delegiert
				\begin{itemize}
					\item \alert<1-> {Observer}
					\item \alert<1-> {Command}
					\item \alert<1-> {Visitor}
					\item Strategy			
					\item Mediator
					\item Iterator
					\item Memento 
					\item State
					\item Chain of Responsibility
				\end{itemize}	
	\end{block}
\end{frame}
